\documentclass[a4paper]{article}
\usepackage[T1]{fontenc}
\usepackage[utf8]{inputenc}
\usepackage{csquotes}
\usepackage{graphicx}
\usepackage{booktabs}
\usepackage{multirow}
\usepackage{microtype}
\usepackage{quoting}
\usepackage{siunitx}
\usepackage[backend=biber,style=alphabetic]{biblatex}
\addbibresource{notes_bibliography.bib}

\DefineBibliographyStrings{english}{%
  references = {Furriadroxus bibliogràficus},
}
\renewcommand{\tablename}{Tauledda}
\renewcommand{\figurename}{Figura}

\begin{document}
\title{Anàlisi de is dadus de is imitiduras de particulau in Sardìnnia}
\maketitle

In su 2020, s' arrelata de s' Agentzia po s' Amparu de s' Ambienti (ARPA) at amostau ca ddoi funt unas cantu chistionis chi pertocant sa calidadi de s' àiri in Sardìnnia~\cite{ARPASardegna2021}. In custu traballu, eus a biri is dadus de su \emph{particulau}, chi funt partixeddas piticas meda chi si podint arrespirai e amuntonai in is prumonis, fendi-ddi dannu mannu~\cite{Davidson2005}. In sa citadi metropolitana de Casteddu, is misuras de custus particulaus funt prus artas de su liminàrgiu sceberau de s' Organitzatzioni Mondiali de sa Saludi (OMS), mancai no propassint su chi at sceberau s' Itàlia. Sa tauledda~\ref{tab:particulate-limits} amostat ca is liminàrgius de s' OMS e de s' Itàlia funt diferentis.

\begin{table}[tbp]
    \centering
    \caption{Liminàrgius de imitidura de particulaus po s' OMS e po sa lei italiana. \label{tab:particulate-limits}}
    \begin{tabular}{clcc}
        \toprule
        \multicolumn{2}{c}{Liminàrgius} & PM 2,5                              & PM 10                                                   \\
        \midrule
        \multirow{2}*{OMS}              & Mèdia a s' annu                     & $\SI{10}{\micro\gram/m^3}$ & $\SI{20}{\micro\gram/m^3}$ \\
                                        & Mèdia a sa dii                      & $\SI{25}{\micro\gram/m^3}$ & $\SI{50}{ug/m^3}$          \\
        \cmidrule(lr){1-4}
        \multirow{3}*{Itàlia}           & Mèdia a s' annu                     & $\SI{25}{\micro\gram/m^3}$ & $\SI{40}{\micro\gram/m^3}$ \\
                                        & Mèdia a sa dii                      &                            & $\SI{50}{\micro\gram/m^3}$ \\
                                        & Propassamentus de sa mèdia a sa dii &                            & $35$                       \\
        \bottomrule
    \end{tabular}
\end{table}

\section{Daily measurements}




\section{Is contus de su 2010}

\begin{figure}[tbp]
    \centering
    \includegraphics[width=\columnwidth]{./figures/pm25.png}
    \caption{imitidura de PM2.5 in sa citadi metropolitana Casteddu.}
    \label{fig:pm25-prov-casteddu}
\end{figure}

\begin{figure}[tbp]
    \centering
    \includegraphics[width=\columnwidth]{./figures/pm10.png}
    \caption{imitidura de PM10 in sa citadi metropolitana  de Casteddu.}
    \label{fig:pm10-prov-casteddu}
\end{figure}

\printbibliography

\end{document}